\documentclass[a4paper,10pt]{book}
\usepackage[utf8]{inputenc}

%----------------------------------------------------------------------------------------
%	TITLE PAGE
%----------------------------------------------------------------------------------------

%\newcommand*{\plogo}{\fbox{$\mathcal{PL}$}} % Generic publisher logo

\newcommand*{\titleGM}{\begingroup % Create the command for including the title page in the document
\hbox{ % Horizontal box
\hspace*{0.2\textwidth} % Whitespace to the left of the title page
\rule{1pt}{\textheight} % Vertical line
\hspace*{0.05\textwidth} % Whitespace between the vertical line and title page text
\parbox[b]{0.75\textwidth}{ % Paragraph box which restricts text to less than the width of the page

{\noindent\Huge\bfseries REST API \\documentation}\\[2\baselineskip] % Title
{\large \textit{\date{}}}\\[4\baselineskip] % Tagline or further description
{\Large \textsc{Orlin Gueorguiev}} % Author name

\vspace{0.5\textheight} % Whitespace between the title block and the publisher

%{\noindent The Publisher \plogo}\\[\baselineskip] % Publisher and logo
}}
\endgroup}

\newcommand{\thead}[1]{\textbf{\large{#1}}}

\begin{document}

\titleGM 

\tableofcontents

\chapter{General concepts}
\label{chap:general}

TODO

\chapter{Rest Calls}
\label{chap:restCalls}

\section{Default search operations}
\label{sec:defaultsearch}

\begin{tabular}{l | c | l}
\thead{Param} & \thead{Default value} & \thead{Description}\\
\hline
pagination & 10 & How many elements should be shown
startFrom & 0 & Which should be the first element to be shown, counting starts from 0
 
\end{tabular} 
\section{Country}
\label{sec:country}

\subsection{Create a country}
\label{s:ccountry}

\begin{tabular}{l | c | l}
\thead{Param} & \thead{Optional} & \thead{Description}\\
\hline
id & true & Support the id, if you wish to update an already existing entity\\
name & false & The name of the country \\
zipCodePrefix & false & The zip code prefix, for example DE for Germany \\
isoCountryCode & false & The ISO country code, for example D for Germany
 
\end{tabular} 




\end{document}
