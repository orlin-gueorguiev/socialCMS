\documentclass[a4paper,10pt]{report}
\usepackage[utf8]{inputenc}

%----------------------------------------------------------------------------------------
%	TITLE PAGE
%----------------------------------------------------------------------------------------

%\newcommand*{\plogo}{\fbox{$\mathcal{PL}$}} % Generic publisher logo

\newcommand*{\titleGM}{\begingroup % Create the command for including the title page in the document
\hbox{ % Horizontal box
\hspace*{0.2\textwidth} % Whitespace to the left of the title page
\rule{1pt}{\textheight} % Vertical line
\hspace*{0.05\textwidth} % Whitespace between the vertical line and title page text
\parbox[b]{0.75\textwidth}{ % Paragraph box which restricts text to less than the width of the page

{\noindent\Huge\bfseries REST API \\documentation}\\[2\baselineskip] % Title
{\large \textit{\date{}}}\\[4\baselineskip] % Tagline or further description
{\Large \textsc{Orlin Gueorguiev}} % Author name

\vspace{0.5\textheight} % Whitespace between the title block and the publisher

%{\noindent The Publisher \plogo}\\[\baselineskip] % Publisher and logo
}}
\endgroup}

\newcommand{\thead}[1]{\textbf{\large{#1}}}

\begin{document}

\titleGM 

\tableofcontents

\chapter{General concepts}
\label{chap:general}


\chapter{Rest Calls}
\label{chap:restCalls}

\section{General operations}
\label{sec:defaultsearch}


\subsection{General search paramethers}
\label{s:defsearchparams}
\begin{tabular}{l | c | l}
\thead{Param} & \thead{Default value} & \thead{Description}\\
\hline
pagination & 10 & How many elements should be shown\\
startFrom & 0 & Which should be the first element to be shown, counting starts from 0
\end{tabular}
All search paramethers are optional.

\textbf{Example:} \textit{socialCRMRest/country?startFrom=45\&pagination=15}

\subsection{Get by id}
\label{s:getById}
We use the standard REST notation: \textit{\textless{}entity\textgreater/id}\\
\textbf{Example:} \textit{socialCRMRest/country/15}

\subsection{Delete by id}
\label{s:deleteById}
We use the standard REST notation: \textit{\textless{}entity\textgreater/id}\\
\textbf{Example:} \textit{socialCRMRest/country/15}


\section{Country}
\label{sec:country}

\subsection{Create a country}
\label{s:ccountry}

\begin{tabular}{l | c | l}
\thead{Param} & \thead{Optional} & \thead{Description}\\
\hline
id & true & Support the id, if you wish to update an already existing entity\\
name & false & The name of the country \\
zipCodePrefix & false & The zip code prefix, for example DE for Germany \\
isoCountryCode & false & The ISO country code, for example D for Germany
 
\end{tabular} 

\section{Address}
\label{sec:address}

\subsection{Create an address}
\label{s:caddress}

\begin{tabular}{l | c | l}
\thead{Param} & \thead{Optional} & \thead{Description}\\
\hline
id & true & Support the id, if you wish to update an already existing entity\\
countryId & false & The id of the country, where the address is\\
zipCode & false & The zip code of the area, where the address is \\
street & false & The street of the address \\
 
\end{tabular} 


\section{Web Presence}
\label{sec:WebPresence}

Local path is \textbf{/webPresence}.

\subsection{Create a Web Presence}
\label{s:cWebPresence}

\begin{tabular}{l | c | l}
\thead{Param} & \thead{Optional} & \thead{Description}\\
\hline
id & true & Support the id, if you wish to update an already existing entity\\
webpage & true & The webpage of the entity\\
email & true & The email of the entity \\
 
\end{tabular} 

\section{Person}
\label{sec:Person}

\subsection{Create a Person}
\label{s:cPerson}

\begin{tabular}{l | c | l}
\thead{Param} & \thead{Optional} & \thead{Description}\\
\hline
id & true & Support the id, if you wish to update an already existing entity\\
firstName & true & First name of the person\\
lastName & true & Last name of the person\\
webPresenceId & true & The id of the webpresence of the person\\
addressId & true & The persons' address\\
companyId & true & The company, where the person works\\
 
\end{tabular} 


\section{Company}
\label{sec:Company}

\subsection{Create a Company}
\label{s:cCompany}

\begin{tabular}{l | c | l}
\thead{Param} & \thead{Optional} & \thead{Description}\\
\hline
id & true & Support the id, if you wish to update an already existing entity\\
name & false & The company name\\
webPresenceId & true & The company webpage details\\
addressId & true & The company address\\
 
\end{tabular} 

\end{document}
